\documentclass{article}
\usepackage{tikz}
\usetikzlibrary{arrows.meta, shapes.geometric, calc, shadows}
\tikzstyle{square} = [regular polygon,regular polygon sides=4]

\begin{document}
\section{Flooding algorithm}
For Equations
\begin{equation}
  \label{eq:cut-1}
  m_{1}x+m_{2}y+m_{3}z = \alpha
\end{equation}
Supposed it ranks as $m_{1}<m_{2}<m_{3}$.

Three intercepts on the coordcates are
\begin{equation}
  \label{eq:cut-1}
  m_{1}x+m_{2}y+m_{3}z = \alpha
\end{equation}

\begin{equation}
  \begin{aligned}
  \label{eq:cut-1}
  h_{1} = \frac{\alpha}{m_{1}} \\
  h_{2} = \frac{\alpha}{m_{2}} \\
  h_{3} = \frac{\alpha}{m_{3}}
  \end{aligned}
\end{equation}

where $h_{1} > h_{2} > h_{3}$.

Based on the 

\subsection{Cut 1}

In this case, $\alpha < m_{1}$, so that $h_{i}<1$, the cutting polyhedron ia a tetrahedron.

\begin{equation}
\begin{aligned}
\label{eq:cut1}
V &= \frac{1}{6}h_{1}h_{2}h_{3} \\
& = \frac{\alpha}{6m_{1}m_{2}m_{3}}
\end{aligned}
\end{equation}

\begin{equation}
\begin{aligned}
\label{eq:cut1}
  c_{i}= \frac{1}{4}h_{i} = \frac{\alpha}{m_{i}}\\
\end{aligned}
\end{equation}

\subsection{Cut 2}

In this case, $m_{1} \le \alpha < m_{2}$, $h_{1}>1$, $h_{2}<1$ and $h_{3}<1$.
The cutting  plane is a quadrilateral and the polyhedron can be expressed
as a larger tetrahedron minus a smaller tetrahedron

\begin{equation}
\begin{aligned}
\label{eq:cut1}
V &= \frac{1}{6}h_{1}h_{2}h_{3} \left ( 1 - \frac{(h_{1}-1)^{3}}{h_{1}^{3}} \right )\\
&= \frac{1}{6}h_{1}h_{2}h_{3} \frac{3h_{1}^{2} - 3h_{1} + 1}{h_{1}^{3}} \right )\\
& = \frac{h_{2}h_{3}}{6h_{2}^{2}} + \frac{(h_{1}-1)h_{2}h_{3}}{h_{1}}\\
& = \frac{\alpha}{6m_{1}m_{2}m_{3}}
\end{aligned}
\end{equation}

\begin{equation}
\begin{aligned}
\label{eq:cut1}
V &= \frac{1}{6}h_{1}h_{2}h_{3} \left ( 1 - \frac{(h_{1}-1)^{3}}{h_{1}^{3}} \right )\\
&= \frac{1}{6}h_{1}h_{2}h_{3} \frac{3h_{1}^{2} - 3h_{1} + 1}{h_{1}^{3}} \right )\\
& = \frac{h_{2}h_{3}}{6h_{1}^{2}} + \frac{(h_{1}-1)h_{2}h_{3}}{h_{1}}\\
& = \frac{m_{1}}{6m_{2}m_{3}} + \frac{\alpha(\alpha-m_{1})}{m_{2}m_{3}}\\
\end{aligned}
\end{equation}

\begin{equation}
\begin{aligned}
\label{eq:cut1}
V &= \frac{1}{6}h_{1}h_{2}h_{3} \left ( 1 - \frac{(h_{1}-1)^{3}}{h_{1}^{3}} \right )\\
&= \frac{1}{6}h_{1}h_{2}h_{3} \frac{3h_{1}^{2} - 3h_{1} + 1}{h_{1}^{3}} \right )\\
& = \frac{h_{2}h_{3}}{6h_{1}^{2}} + \frac{(h_{1}-1)h_{2}h_{3}}{h_{1}}\\
& = \frac{m_{1}}{6m_{2}m_{3}} + \frac{\alpha(\alpha-m_{1})}{m_{2}m_{3}}\\
\end{aligned}
\end{equation}

\section{Advection of Volume function}
Use a simple 3 stencil 1D grid as an example:

\begin{figure}
\begin{tikzpicture}
\pgfmathsetmacro{\xo}{0}
\pgfmathsetmacro{\yo}{0}
\draw[step=1, gray, very thin, xshift = \xo cm, yshift= \yo cm] (0,0) grid (3,1);
\end{tikzpicture}
\end{fugire}


\subsection{Weymouth-Yue}
Original volume plus the boundary flux (Eulerian).
\begin{equation}
  \label{eq:wy}
  \tilde{f_{c}} = f_{c} + VOF^{1}_{c} - VOF^{3}_{c} - VOF^{1}_{r} + VOF^{3}_{l}
\end{equation}

\subsection{CIAM}
Backward lagrangian of the grid face and find the intersection between
two faces (Lagrangian).

\begin{itemize}
  \item 
\end{itemize}

\begin{equation}
  \label{eq:CIAM}
  \tilde{f_{c}} = VOF^{2}_{c} + VOF^{1}_{r} + VOF^{3}_{l}
\end{equation}

Compared with W-Y advection, we obtain

\begin{equation}
  \label{eq:CIAM}
  VOF2_{c} = f_{c} - 2 VOF^{1}_{r} - VOF^{1}_{c} - VOF^{3}_{c}
\end{equation}


\end{document}